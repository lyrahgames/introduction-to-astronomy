\documentclass[a4paper,9pt]{article}
\usepackage{extsizes}
\usepackage[left=25mm,right=25mm,top=30mm,bottom=30mm]{geometry}
\pagestyle{empty}
\linespread{0.5}\selectfont
\usepackage[utf8]{inputenc}
\usepackage[T1]{fontenc}
\usepackage{times}
\usepackage{microtype}
\usepackage[ngerman]{babel}
\usepackage{multicol}
\usepackage{titlesec}
\titleformat*{\section}{\bfseries}

\begin{document}
  \hrule
  \begin{center}
    \scshape
    \huge
    Einführung in die Astronomie \\[\baselineskip]
    Gliederung \\ [\baselineskip]
    \normalfont
    \normalsize
    \begin{minipage}[c]{0.49\textwidth}
      Markus Pawellek \\[\baselineskip]
      markuspawellek@gmail.com
    \end{minipage}
    \hfill
    \today
  \end{center}
  \hrule
  \bigskip
  \begin{multicols}{3}
    \section{Astronomie als Wissenschaft} % (fold)
    \label{sec:astronomie_als_wissenschaft}
      \begin{enumerate}
        \item Objekte der Astronomie
        \item Methoden der Astronomie
        \item Unterteilung der Astronomie
        \item Astronomie und andere Wissenschaften
      \end{enumerate}
    % section astronomie_als_wissenschaft (end)

    \section{Astronomische Beobachtungen und Instrumente} % (fold)
    \label{sec:astronomische_beobachtungen_und_instrumente}
      \begin{enumerate}
        \item Beobachtungen
        \item Teleskope
        \item Leistungsvermögen der Teleskope
        \item Detektoren
        \item Montierungen
      \end{enumerate}
    % section astronomische_beobachtungen_und_instrumente (end)

    \section{Sphärische Astronomie} % (fold)
    \label{sec:sphärische_astronomie}
      \begin{enumerate}
        \item Himmelskugel und Sternbilder
        \item Sphärische Trigonometrie
        \item Definitionen
        \item Koordinatensysteme
        \item Transformationen zwischen den Systemen
        \item Tägliche Bewegung der Sterne
        \item Jährliche Bewegung der Sonne
      \end{enumerate}
    % section sphärische_astronomie (end)

    \section{Zeit und Kalendar} % (fold)
    \label{sec:zeit_und_kalendar}
      \begin{enumerate}
        \item Sternzeit
        \item Sonnenzeit
        \item Zusammenhang
        \item Ortszeiten, Weltzeiten und Zonenzeit
        \item Kalendar
      \end{enumerate}
    % section zeit_und_kalendar (end)

    \section{Astrometrie} % (fold)
    \label{sec:astrometrie}
      \begin{enumerate}
        \item Koordinatensysteme und Bezugssysteme
        \item Refraktion
        \item Aberrationen
        \item Parallaxe
        \item Präzession und Nutation
        \item Eigenbewegung der Sterne
      \end{enumerate}
    % section astrometrie (end)

    \section{Himmelsmechanik} % (fold)
    \label{sec:himmelsmechanik}
      \begin{enumerate}
        \item Problemstellung
        \item Bewegungsgleichungen
        \item Drehimpulserhaltung
        \item Energieerhaltung
        \item Laplace-Integral
        \item Geometrie der Bahnen
        \item Keplersche Gesetze
      \end{enumerate}
    % section himmelsmechanik (end)

    \section{Astrophotometrie} % (fold)
    \label{sec:astrophotometrie}
      \begin{enumerate}
        \item Grundbegriffe
        \item Scheinbare Helligkeit
        \item Absolute Helligkeit
        \item Schwarzer Strahler
        \item Plancksches Gesetz
        \item Näherungen
        \item Wiensches Verschiebungsgesetz
        \item Stefan-Boltzmann-Gesetz
      \end{enumerate}
    % section astrophotometrie (end)

    \section{Astrospektroskopie} % (fold)
    \label{sec:astrospektroskopie}
      \begin{enumerate}
        \item Absorption und Emission der Strahlung
        \item Typen von Spektren
        \item Doppler-Effekt
      \end{enumerate}
    % section astrospektroskopie (end)

    \section{Das Sonnensystem} % (fold)
    \label{sec:das_sonnensystem}
      \begin{enumerate}
        \item Überblick
        \item Planeten
        \item Kleinkörper
        \item Extrasolare Planetensysteme
      \end{enumerate}
    % section das_sonnensystem (end)

    \section{Die Sonne} % (fold)
    \label{sec:die_sonne}
      \begin{enumerate}
        \item Beobachtungsdaten
        \item Spektrum
        \item Aufbau und Energiequelle
      \end{enumerate}
    % section die_sonne (end)

    \section{Sterne} % (fold)
    \label{sec:sterne}
      \begin{enumerate}
        \item Allgemeines
        \item Kenngrößen
        \item Spektralklassifikation
        \item Hertzsprung-Russel-Diagramm
      \end{enumerate}
    % section sterne (end)

    \section{Sternentwicklung} % (fold)
    \label{sec:sternentwicklung}
      \begin{enumerate}
        \item Modelle
        \item Entstehung
        \item Protostern
        \item Hauptreihe
        \item Roter Riese
        \item Endstadien
      \end{enumerate}
    % section sternentwicklung (end)

    \section{Ungewöhnliche Sterne} % (fold)
    \label{sec:ungewöhnliche_sterne}
      \begin{enumerate}
        \item Doppelsterne
        \item Eruptionsveränderliche
        \item Pulsationsveränderliche
        \item Novae und Supernovae
        \item Kompakte Sterne
      \end{enumerate}
    % section ungewöhnliche_sterne (end)

    \section{Die Milchstraße} % (fold)
    \label{sec:die_milchstrasse}
      \begin{enumerate}
        \item Bestandteile und Aufbau
        \item Rotation, Masse und Spiralstruktur
        \item Sternhaufen
      \end{enumerate}
    % section die_milchstraße (end)

    \section{Galaxien} % (fold)
    \label{sec:galaxien}
      \begin{enumerate}
        \item Entfernungsbestimmung
        \item Klassifikation
        \item Galaxien mit aktiven Kernen
        \item Verteilung
      \end{enumerate}
    % section galaxien (end)
  \end{multicols}
\end{document}